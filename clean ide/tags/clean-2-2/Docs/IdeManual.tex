% To Do:
% � use consistent formatting for Menu options (and file names etc.)
% � Time and Heap profiler sections
% � commandline arguments section
% � dynamics section
% � slibs section
% � dlibs section
% � batchbuild
% � precompile/postlink
% � building compliant Windows apps
% � building compliant Macintosh apps

\documentclass[a4paper]{article}
%\usepackage{textcomp}
\usepackage[T1]{fontenc}
\usepackage{ae}
%\usepackage{pslatex}
%\usepackage{mathptm}
%\usepackage{pifont}
\usepackage{multicol}
\usepackage{geometry}
\usepackage{fancyhdr}
\usepackage{graphicx}
\usepackage{fancyvrb}
%\usepackage{makeidx}
\usepackage[toc]{multitoc}
\usepackage{longtable}
\usepackage{array}
%\usepackage{float}
\usepackage{fancybox}
\usepackage{color}
\usepackage{colortbl}
%\usepackage{url}
%\usepackage{citeref}
\usepackage{tocloft}
\usepackage[colorlinks,breaklinks,pdfstartview={FitH},pdfpagemode=None,pdftex]{hyperref}
\usepackage{amssymb}	%for \backprime
%\newenvironment{mylist}
% {
%\begin{Pilist}{pzd}{113}
% }
% {
%\end{Pilist}
% }

\newenvironment{menu}
{\begin{description}}
{\end{description}}

\newcommand \note[1]{\leavevmode\marginpar{100}\textsc{#1}}		% wil dit in marge en alleen wanneer draft-optie...
\newcommand \inref[1]{\nameref{#1}, p.\pageref{#1}}
\newcommand \screen[1]{\par\fbox{\scalebox{.5}{\includegraphics{#1}}}\par\noindent}
%\nameref{fig:filemenu}
%\begin{figure}[H]
%\centering
%\caption[blah blah]{The File Menu}
%\label{fig:filemenu}
%\fbox{\includegraphics{filemenu}}
%\end{figure}

\newcommand \light{\rowcolor[gray]{.9}}
\newcommand \dark{\rowcolor[gray]{.6}}

\newcommand \keyword[1]{{\textbf{#1}}}

\DefineShortVerb{\|}

\newcolumntype{L}{<{$}l>{$}}

\definecolor{mygray}{gray}{0.75}

\hypersetup{pdftitle=Clean System User Manual}

\title{{%
	Clean System User Manual\\
	%Draft Version 1.4.00\\
	\textcopyright 1987-2001
}}

\author{D.F.R. van Arkel}

\setcounter{tocdepth}{2}

%\makeindex

%\makeatletter
%\setlength{\cftbeforepartskip}{1.0em \@plus\p@}
%\makeatother

\pagestyle{fancy}
\cfoot{}
\rfoot{\thepage}
\lhead{\slshape \leftmark}

\begin{document}
\maketitle
\thispagestyle{empty}

Clean is a general purpose, higher order, pure and lazy functional
programming language that can be used for the development of  real world applications. 
The Clean System is a special programming
environment for Clean.

\tableofcontents

\section*{Copyright Notice and Disclaimer}
\begin{center}
\shadowbox{
\begin{tabular}{c}	%{|c|}
	%\hline
	Clean and the Clean System 
	\textcopyright 1987-2001\\
	
	Computing Science Institute\\
	University of Nijmegen\\
	Nijmegen - The Netherlands\\
	and\\
	HILT - High Level Software Tools B.V.\\
	Nijmegen - The Netherlands\\
	\url{mailto:clean@cs.kun.nl}\\
	
	We do not make any warranty, either express or implied, with respect to this 
	software.%\\
	%\hline
\end{tabular}
}\end{center}

\section{Availability}\label{availability}

Clean and the Clean System can be used free of charge. They can be obtained
\begin{itemize}
\item via the World Wide Web: \url{http://www.cs.kun.nl/~clean}
\item via ftp: \url{ftp://ftp.cs.kun.nl/pub/Clean}
\end{itemize}
Clean is available for several platforms. Please check the Clean website for the
latest news. If you have any questions, bugs to report, or when you have wishes or
comments please send an electronic mail to
\url{mailto: clean@cs.kun.nl}.


\section{More information on the language Clean}

A description of the syntax and semantics of Clean can be found 
in the following manual which is included in the distribution:

\begin{center}
\shadowbox{
\begin{tabular}{c}
Concurrent Clean Language Report\\
%--- version 1.3 ---\\
Rinus Plasmeijer and Marko van Eekelen\\
University of Nijmegen
\end{tabular}}
\end{center}

\noindent
The ideas behind Clean and its implementation on sequential and parallel 
machine architectures are explained in detail in the following textbook:

\begin{center}
\shadowbox{
\begin{tabular}{c}
Functional Programming and Parallel Graph Rewriting\\
Rinus Plasmeijer and Marko van Eekelen\\
Addison-Wesley Publishing Company\\
\textcopyright 1993\\
ISBN 0-201-41663-8
\end{tabular}}
\end{center}

\noindent
An introduction to functional programming in Clean for undergraduate students is also available. 
Special about the book is that it contains non trivial applications such that the book will also be 
very useful for people who want to develop serious programs:
\begin{center}
\shadowbox{
\begin{tabular}{c}
Functional Programming in Clean\\
Paul de Mast (Polytechnical University Brabant)\\
Jan-Marten Jansen and Dick Bruin (Polytechnical University Leeuwarden)\\
Jeroen Fokker (University of Utrecht)\\
Pieter Koopman, Sjaak Smetsers, Marko van Eekelen and Rinus Plasmeijer (all University of Nijmegen)
\end{tabular}}
\end{center}

\noindent
The books and the latest research papers on Clean can be found via the Clean home page at \url{http://www.cs.kun.nl/~clean}.

\section{The Clean System}
The Clean System is a programming 
environment for Clean written in Clean itself.
The current version of the system consist of:

\begin{itemize}
\item  The Project Manager: an automatic make facility for Clean projects
\item  The Text Editor: a text editor with Clean sensitive search facilities and syntax colouring
\item  The Clean Compiler: compiler for the platform on which the system is running
\item  The Code Generator: code generator for the platform on which the system is running
\item  The Linker: generates stand-alone (Mac or Windows) applications
\item  Libraries for arithmetic, I/O handling, utility modules for interfacing
\item The Time Profiler: for viewing time profiles
\item The Heap Profiler: for viewing heap profiles
\item Sparkle: a theorem prover for Clean programs, see \url{http://www.cs.kun.nl/Sparkle}
%\item Visual Editor: a visual editor for Clean Object I/O programs
\end{itemize}
The project manager and editor are integrated into the Clean Integrated Development Environment (IDE). The remaining tools
are separate applications automatically launched by the Clean Ide when this is required.

\section{Installation}\label{installation}
\begin{itemize}
\item Make a backup of your old Clean programs, Clean system and I/O libraries.
\item Unpack the Clean archive to a suitable directory on your hard disk (i.e.
|C:\Clean\|).
\item The Clean System generates native applications. If you have virus 
detectors they may complain about this. Clear the protection for the Clean 
compiler (\verb+cocl.exe+) and editor (\verb+CleanIDE.exe+). They should be allowed
to create, edit and delete files . Make sure that you have an original unaffected 
Nijmegen version of the Clean System (see~\inref{availability}).
\item The first time you launch the Clean IDE it will ask if you want to allow it to install
the Clean IDE into the Windows registry. This binds the various file types used by the Clean
System so that they have nice icons and so that double-clicking them will launch the appropriate
tool.
\end{itemize}
\section{Uninstalling}\label{uninstall}
%\note{Need something here about de-installation...}
To uninstall Clean: if you allowed the Clean IDE to enter its filetypes and settings into the
Windows registry you might want to remove them again. Choose ``IDE Options...''' from the Defaults
menu and press the ``Clear Registry'' button, then quit the IDE. You can now remove Clean by moving
the Clean folder to the Trash.
\section{Quick Start}\label{quickstart}
\begin{itemize}
\item Start the Clean IDE (|CleanIDE.exe|). You will see an empty project window. First we need
to make a new project or open an existing project. A Clean project consists of definition modules
(|.dcl| files) and implementation modules (|.icl| files). A definition module  indicates which definitions
defined in the corresponding implementation module are exported. The topmost root module (the
main module) is always an implementation module.
\screen{a_openfile}
\item Open the main module of the ``Game of Life'', |LifeGameExample.icl| (in the |Examples\ObjectIO Examples|
|\Life\| directory).
The source text of this Clean implementation module will be displayed in a text window and can be modified with the built-in editor.
%\screen{a_textwindow}
\screen{openfile}
\item We now make a new Clean project which will have LifeGameExample as its main module by selecting
New Project from the File menu. This presents a dialogue allowing you to set the project name and location.
It is easiest to simply use the suggested defaults. As the examples come with pre-setup projects the IDE will
ask wether you want to replace the existing version. Selecting yes is safe as you will exactly recreate the existing
project in the following steps.
\screen{a_newproject}
The Clean IDE now knows where to save the project administration and the name of the main module
(LifeGameExample) will appear in the project window.
\item The LifeGameExample uses other Clean modules, namely modules from the StdEnv library and modules from
the Object I/O library. Each project has an environment in which it is specified which libraries are being used
in the project.
\screen{a_chooseenv}
The Clean IDE has a number of popular environments predefined. You have to choose which one you
would like to use in this particular project. From the Environment menu choose Object IO. Now the IDE knows which
libraries to use in this project.
\screen{env_paths}
By the way, one can define new environments (``New Environment...'') or change existing ones (``Edit Current...''). 
One can set the libraries to be used, by adding the paths of the folders in which the corresponding Clean source 
modules can be found. In the environments one can also include ``real'' libraries containing object code (Static 
Libraries and Dynamic Libraries). One can even define which IDE components (Tools) to use (which Clean 
compiler, e.g. version 1.3 or version 2.0 , which linker, and so on). In this way you can easily assign your own 
preferences to a project or switch between settings just by choosing another environment setting. The environments 
will be remembered by the IDE, the chosen environment will be remembered in the project administration file.
In addition to the environment you might also want to use more project specific libraries which you might have 
created to structure your projects, but, which are too specific to put in an environment. For example, the Life 
Game has a help facility. The corresponding Clean modules are put in a separate folder to be shared by other 
examples.
\screen{a_projectoptions}
\item Choose ``Project Options...'' from the ``Project'' menu, select the ``Paths'' panel and add the library 
|ObjectIO Examples\gui utilities| to the Life Game project paths.
\screen{projectpaths}
Now all libraries used by the ``LifeGameExample'' are included and we can generate the application. 
Of course, next time, when you open the Life Game project, everything is remembered.
\screen{update_run}
\item From the ``Project'' menu choose ``Update and Run'' . All Clean modules used by ``LifeGameExample'' 
will appear in the project window. Those modules which are not up to date will be compiled to abc code. If 
no errors are found, the code generator will translate all abc-files to object code. Finally, the linker will build 
the application. If everything is fine, the Clean IDE will launch the application automatically.
\screen{lifegame}
\end{itemize}

Building the application might not succeed when you have a virus detector 
active (see~\inref{installation}). Launching will not succeed when the generated application
does not have enough memory to run. 
You can change the amount of allocated memory in the Project Options dialogue (see~\inref{project_options}).

\section{Projects}
The project manager in the Clean IDE checks and controls the consistency of all 
the files, windows and options of the system such that  the creation and
execution of Clean applications can be invoked with only 
one command.

All information about a project is remembered such that a second compilation 
goes much faster than the first one. The information about a project is stored in 
a file (\verb+name.prj+) such that projects can be continued in the same status they are 
left.

A Clean program consists of modules that can be compiled separately. There are 
implementation modules (\verb+name.icl+) and definition modules (\verb+name.dcl+). Only 
implementation modules can be compiled explicitly, the corresponding 
definition modules are checked automatically.

Modules are hierarchical. The topmost implementation module is variously called the 
main, start, or root module. It should contain a Start rule (see the Clean 
Language Report).

The project manager needs to know what the main module is. When this 
module is known all other modules (and the corresponding intermediate files) 
in the hierarchy are examined by the manager. All these files together are 
called a project (see~\inref{files}). The default name of the project is the name of the main 
module. The settings of a particular project are stored in the project file.

The project manager will scan your disk to find the files it needs. In order to 
restrict the search time the search paths should be set as accurately as possible 
(see~\inref{paths}).

A new project can be created by selecting |New Project...| in the |File| menu.
The default name of the project is the name of the module in the active editor window 
and the default main module is also that module.
Any implementation
module can be assigned to be the main module via |Set Main Module| in the |Project| menu. 

A window can be opened on a file containing a Clean module. The contents can 
be changed with the text editor (see~\inref{text_editor}). When you generate an 
application, the project manager may warn you when windows and files have 
different contents. The Clean compiler always uses the version on disk, so don't 
forget to save your changes before you bring your project up-to-date.

\section{Modules}

\subsection*{Implementation modules}
An implementation module (\verb+name.icl+) consists of a sequence of type, function, 
graph, class, macro and other definitions. These definitions have in principle 
only a local scope: they only have meaning in the implementation module they 
are defined in. In the corresponding definition module one can specify which 
types and rules (defined in the corresponding implementation module) are 
exported, i.e. are made visible to the outside world.

\subsection*{The main module}
The main module is always an implementation module. In its module header 
the keyword \keyword{implementation} can be left out. In that case the main module 
does not need a corresponding definition module. Furthermore, a main module 
should always contain an explicit or implicit declaration of the Start rule, i.e. 
this rule should either be defined in the main module or
imported from another module. Any implementation module can become a 
main module by inserting a Start rule. This is helpful for the debugging of 
modules.

\subsection*{Definition modules}
In a definition module (\verb+name.dcl+) one can specify which definitions (defined in 
the corresponding implementation module) are exported, i.e. are made visible 
to the outside world. All other definitions given in the implementation module 
remain hidden.

\subsection*{System modules}
System modules are special definition modules indicated by the keyword 
\keyword{system} instead of \keyword{definition}. This indicates that the corresponding
implementation modules are special system modules. These modules are predefined 
and can only be made by system programmers.

\section{Files}\label{files}

Files used by the Clean System have certain extensions that indicate 
their use. The following kinds of files are recognized by the Clean 
System:
\begin{description}
\item[name.icl] A Clean implementation module with the name `name'.
\item[name.dcl] A Clean definition module.
It is assumed that a \verb+.icl+ file and its corresponding \verb+.dcl+ file are always kept in 
one and the same folder.

\item[name.abc] An abc file generated by the Clean compiler.
\item[name.o] An object file generated by the Clean code generator.
\item[name.s] An assembly file generated by the Clean code generator.

\noindent
The generated abc and object files are stored in the Clean System Files folder 
that will be generated in the folder where the corresponding \verb+.icl+ (and \verb+.dcl+) files 
are stored.

\item[name.prj] The project file.

\item[name.exe] An application generated by the linker.
The generated application will be stored in the location given in the Project Options
dialogue reachable from the \verb+Project+ menu. By default it uses the same name as the project
with \verb+.exe+ extension and is stored in the same folder as the project file. 
\end{description}
The \verb+.icl+, \verb+.dcl+, \verb+.abc+, \verb+.s+ files are text files which can be edited with the 
Clean editor.

\section{Windows}

Several kinds of windows can appear on your screen.
\begin{description}
\item   [.icl]
An editor window on an implementation module.

\item   [.dcl]
An editor window on a definition module.

\item   [Types]
A window that displays derived and checked type information as well as
the derived and assumed strictness information.

\item   [Errors \& Warnings]
A window that displays warnings and error messages generated by the system.

\item   [Project]
A window that displays information on the current project.

\item [Search Results]
A window displaying the results of the last search command.

\item[Clipboard] A window displaying the current contents of the clipboard.
\item[Console] A window showing the input and output of a Clean console application
(see~\inref{environment_options}).
\end{description}

\noindent
Any other text file can be opened in an editor window as well.

\section{Paths}\label{paths}
In a number of locations paths can be set for the IDE. These determine where the project
manager looks for Clean modules and intermediate files. For optimal performance these should
reflect as accurately as possible where modules can be found. In general the project manager
first searches the paths given in the \verb+Project Options+ dialogue in the order that they
are listed. If the module is not found there then the Environment paths of the current environment
are searched.

\section{IDE Options}\label{ide_options}
\screen{ide_options}
The IDE has a number of global options which can be set from the \verb+IDE Options+ dialogue
available from the \verb+Defaults+ menu.
\begin{description}
\item[Switch windows] This setting influences the way project switches are handled. If it is on
then when switching between projects the IDE closes the windows belonging to the old project and
remembers their settings and then opens the windows belonging to the new project. If this setting
is off then all open windows remain open and no new ones are opened.
\item[Show toolbar] This setting determines wether or not the IDE toolbar is displayed. If it is on
and the Clean IDE can find its bitmaps in the bitmaps folder then a toolbar is displayed the next time
the IDE starts up.
\item[Verbose diagnostics] This activates more verbose diagnostics from the Clean project manager
when updating a project. If you are having difficulties convincing a project to build this can be a useful
option to activate.
\item[AltGr workaround] This activates a dirty hack for some users with non-english keyboards. It disables
all menu shortcuts using symbols instead of letters. French and German users might want to activate this if
they are currently having trouble accessing special symbols on their keyboard. It only has effect when launching
the IDE so to see the result you need to change the flag and then quit and restart the IDE.
\item[Newline Handling / Newline Convention] These settings together determine the newline handling for editor windows.
There are to possible behaviours: one is to maintain the existing newline convention for existing files, the other
is to always use the preferred newline convention. This preferred newline convention is settable, native refers
to the platform native newline convention e.g. Dos when using Windows.
\item[Time Profiler/Heap Profiler/Theorem Prover] These  settings allow customisation of the profiler and prover
locations as new versions appear. Note that this also automatically adjusts the Windows registry settings for
these tools so that double-clicking on profile files activates the correct version of the profiler. Also note that the
current implementation of the file browser dialogue on selecting an application asks wether you want to replace it.
Just answer yes, the application will not actually be replaced. This will be fixed in a future version.
\item[Clear Registry] This button clears all Clean entries from the Windows registry. Use this to remove all
traces of Clean from your system.
\end{description}

\section{Environment Options}\label{environment_options}
Environments are used to gather commonly used settings. The active environment for a project is the environment
that is flagged in the \verb+Environment+ menu when that project is active. You can select a
new active environment by choosing it in the |Environment| menu. 
The environment options are
split into a number of groups. These options
can be changed by selecting \verb+Edit Current...+ from the \verb+Environment+ menu.
\begin{description}
\item	[Tools] 
Here the compiler, code generator, and linkers can be set that are used
within an environment. Also the associated version of the  abc-code has to be given so that
the Project Manager can check that compatible versions of these tools are being used. Most
programmers wil only be using a single version of the Clean compiler and will never need to
change these settings. For people mixing different versions of the Clean compiler the necessary
details will be given seperately.

The ``Console in IDE'' flag activates an experimental feature which redirects the console i/o for a
Clean Console application into a special IDE window.
The ``Sync/Async/Pers'' radiocontrol with associated editbox control the settings for the Clean make facility.
These, together with the other tools options, will be subsumed by a system of tool descriptors in a future
version of the IDE. For now use ABC version 916 together with the Pers flag when using the Clean 2.0
tools, and ABC version 918 together with the Sync flag when using the Clean 1.3.x tools.
\screen{env_tools}

\item	[Paths] Here search paths can be specified which are common to all projects using
this environment. See~\inref{paths}. Up/Dn allows reordering the paths to influence where modules
are found.
\screen{env_paths}

\item	[Dynamic Libraries] Here dynamic libraries can be specified which are not already implicitly included
but are required for a project.

\item	[Static Libraries] Here static libraries can be specified, upcoming versions of the Clean System will
be able to generate libraries of Clean modules.

\item	[Object Modules] Here extra object modules can be specified. These last three options are generally only
needed by Clean wizards.

\end{description}

\section{Project Options}\label{project_options}
\begin{description}
	\item[Application]
	In this dialogue the run-time options for an application can be set. 
	The options are used when a new application is generated.
%	This version of the Clean System is able to generate sequential 
%	applications. We are working on a code generator that produces parallel code.
\screen{application_options}
		\begin{itemize}
		\item   Application Options

		Set the heap space and the maximal amount of space available for the stacks. 
		The stack space is created outside the heap space. Initially not all of the
		heap is used (the Initial Heap Size). If a garbage collection does not free
		enough memory, the heap is expanded (with the factor Next Heap Size Factor)
		up to a maximum (the Maximum Heap Size).
		
		The application will stop with a run-time error message when a heap turns out 
		to be too small during execution.
		After a stack overflow the program may crash or behave unpredictably.
		
		Increase the space sizes when the application needs more memory. It is also 
		possible that the application requires less memory. For an optimal tuning look 
		at the run-time information that the application can produce (see the sction on
		Profiling options).
		
%		\note{Document dynamics flag here...}
		\item Enable dynamics
		Activates support for the new Clean 2.0 language feature of dynamics.
		Essentially it makes sure that all the tools know that a project using dynamics is being
		built so that all the extra steps needed to support them are being taken. As dynamics support
		matures this option will probably develop into its own project type.
		\item Marking Garbage Collector
		When fine-tuning your application performance you can experiment with an alternative
		garbage collector, the marking garbage collector, again check the available run-time
		information to see which setting is optimal.
		
		\item Console
		With the Console options one has control over the way the result of the Start function 
		is displayed.
		
		Choose Show Constructors and all constructors and brackets will be shown. 
		Choose Basic Values Only if you only want to display basic values only (integers, 
		booleans, strings, characters, reals).
		
		Choose No Console if the I/O of the program is handled via the I/O libraries. 
		When this option is chosen no console (output window) for the program will be 
		provided and the result of the Start function will not be displayed. When the 
		Clean program explicitly tries to write to the console a console window will be 
		opened. So, when your program terminates unexpectedly probably a run-time 
		error has been detected. Choose one of the other Output options and run the 
		program again to be able to read the error message.
		
		\item Executable
		Here you can select a different location for the generated executable. You will
		know when you want to do this.
		\end{itemize}
	
		
\screen{profiling_options}
	\item	[Profiling]
		The profiling options can be used to measure the performance of your application.
		With Time profiling enabled, a Clean program will write a profile file after
		execution. This file (\verb+Profile+ on the PowerMac, \verb+ApplicationName Time Profile.pcl+ on Windows) can
		be inspected with the ShowTimeProfile tool. The time profile file is overwritten
		every time you run the profiled application. For the PowerMac you also have to
		specify the target processor. You will get a run-time error message if you choose
		the wrong processor.
		
		Time profiling can also be useful for debugging. With Time Profiling a Clean
		application that unexpectedly halts will print a stack trace of the functions that
		lead to the error. This can be used track down stack overflows, out of bounds
		array indices and `rule does not match' bugs. If you only want stack traces this
		can also be selected here.
		
		With Heap profiling
		the application writes a number of heap files (\verb+ApplicationName Heap Profile0..9+). These files are
		generated when there is a garbage collection and the heap use is above a certain limit
		(the Minimum Profile Heap). The heap files can be inspected with the ShowHeapProfile
		application. 
		
\screen{diagnostics_options}
	\item   [Diagnostics]
		An application can produce all kinds of information to make an optimal tuning 
		of the memory size to be assigned to the application possible.
		
		The execution time will be displayed when the application stops and Show 
		Execution Time is chosen.
		
		Each time the garbage collector is invoked this is displayed when Show Garbage 
		Collections is chosen. The run-time system uses both a copying as well as a 
		mark-scan garbage collection algorithm. It automatically switches between 
		these two garbage collection strategies to obtain an optimal time/space 
		behaviour. This option gives you information about how 
		well the heap size was chosen.
		
		In addition, also the sizes of the different stacks are printed when Print Stack 
		Size is chosen. This helps to tune the sizes of the stacks.
		
		When Write stderr to file is enabled, the messages in the errors window are copied
		to a file with name `Messages' in the current directory.
		
\screen{linker_options}
	\item[Linker Options]
		Here you can select the link method and various link options. 
%		Static linking produces
%		classic applications. Eager and dynamic linking produce applications where parts or all
%		of the application code is not bound until run time. Eager and dynamic linking will be
%		fully supported when the Clean language version 2.0 is released.
		
		Generate Relocations: this is an option for generating relocateable code. It provides
		early support for making Clean DLL's. Only to be used by Clean wizards!
		
		Generate Link Map: this writes out a text file \verb+name.map+ containing the location
		of various Clean symbols in the generated application. Used by Clean wizards to analyze
		low-level application crashes. 
		
		Generate DLL: another option under development similar to the Generate Relocations option.

		Use Standard Runtime Environment: an option for turning of the inclusion of the Clean Runtime
		Environment. Only used by Clean wizards when developing new versions of the runtime environment.
		
		Include Resource Section: a currently Windows only option for copying the resource section from
		another application into the freshly linked application. Used to include icons and whatnot in Windows
		GUI applications. Note that the
		current implementation of the file browser dialogue on selecting an application asks wether you want to replace it.
		Just answer yes, the application will not actually be replaced. This will be fixed in a future version.
% only in Project Options not Defaults...
	\item	[Project Paths] Here the project search paths can be set, see~\inref{paths} and~\inref{environment_options}.
	
	\item	[Extra Objects / Dynamic Libraries / Static Libraries] These are for extra items to be included when linking,
		for more information see~\inref{environment_options}.

\end{description}

\section{Module Options}\label{module_options}
There are also a number of options which can be set per module. These can be set with
the \verb+Module Options+ entry in the \verb+Module+ menu. This displays the module
options dialogue for the actuve window in the editor. If this entry is selected when
the project window is active, then the settings are applied to all modules currently
selected in the project window.
\screen{module_options}
\begin{description}
	\item   [Strictness analyzer]
	When this option is set the strictness analyser will try to determine whether or 
	not the defined functions are strict in their arguments. Strict arguments will be 
	marked with `!' in the type definition.
	
	In general, much better code will be generated when this option is set. The 
	penalty is that the compilation takes more time.
	
	When you are exporting a function with strict arguments, it is for efficiency 
	reasons better to export the strictness information also. You have to do this by 
	hand. Define the type specification including the strictness information in both 
	the definition and the implementation module.
	
	Strictness annotations defined by the programmer are not checked but taken to 
	be true.
	
	\item   [No Types]
	No type or strictness information is displayed when this option is chosen.
	
	\item   [Inferred Types]
	The inferred types of functions that are not typed by the programmer are 
	displayed in the Types window. Strictness information is not displayed.
	
	\item   [Strict Export Types]
	The types of functions that are exported are displayed including inferred 
	strictness information, but only when the type in the definition module does 
	not contain the same strictness information as was found by the strictness 
	analyser. This way it is easy to check for all functions if all strictness
	information is exported.
	
	\item   [All types]
	The types of all functions including strictness information are displayed in the 
	Types window.
	
	\item   [Show Attributes]
	Check this box if you want uniqueness attributes displayed in the listed types.
	
	\item   [Be Verbose]
	If set additional information about the compilation process is given. Turn it off 
	to speed up the compilation.
	
	\item   [Give warnings]
	When this option is set also warnings are produced in the Error window. Turn it 
	on when you are an inexperienced Clean programmer or when you want to be 
	informed about unusual but not erroneous situations.
	
	\item   [Generate comments]
	Set this option if you want to be able to read the generated \verb+.abc+ file. Turn it off 
	to speed up the compilation process.

	\item   [Reuse unique nodes]
	Enables compile-time garbage collection for certain unique nodes. See section 4.5.7 in
	the Clean Language Reference manual for more details.
	
	\item   [Never time profile]
	Don't generate time profiling information for this module, even if time profiling is
	enabled in the \verb+Project Options+ dialogue.
	
	%\item   [Never memory profile]
	%Don't generate memory profiling information for this module, even if memory profiling
	%is enabled in the \verb+Project Options+ dialogue.
\end{description}

\section{The Text Editor}\label{text_editor}

All text files (see~\inref{files}) can be edited with the built-in editor using the commands 
available in the \verb+Edit+ and \verb+Search+ menus.
The editor has an auto indent facility. When this option is on the editor will 
remember the first non-tab, non-space position at the beginning of the line. 
When a return is typed the cursor will move to the remembered position in the 
next (new) line and the needed amount of tabs and spaces will be added.
The auto indent option can be switched on and off in the Editor Settings 
dialogue (see \verb+Editor Settings...+ under the \verb+Defaults+ menu).

The editor also contains two key and mouse combinations with which definitions
of imported Clean symbols (functions, types and constructors) can be 
found:
\verb!Control + Double-Click! on a Clean symbol locates the definition of the symbol 
in its \verb+.dcl+ file.
\verb!Control-Shift + Double-Click! on a Clean symbol locates the definition of the 
symbol in its \verb+.icl+ file.

The same effect can be achieved by selecting a symbol name and then choosing 
\verb+Search Definition+ (for the \verb+.dcl+ file) or \verb+Search Implementation+
(for the \verb+.icl+ file) in the 
\verb+Search+ menu. Warning: Functions that have no type specification may not be 
found. When your Clean modules have a (very) unusual layout the search 
algorithm might also not be able to find the definitions of Clean symbols. 

\section{The Clean Compiler}

The Clean compiler compiles an implementation module to abc-code.
It performs syntactical and semantical checks (see the Clean Language Report). 
The corresponding definition module is also checked, as well as the consistency 
between definition and implementation module. Errors are reported in the 
Errors \& Warnings window.

(Uniqueness) type information is inferred and checked. Strictness information is 
inferred with help of the builtin strictness analyser. On demand, the type and 
strictness information is displayed in the Types window.

If no errors are found, abc-code is generated. For each \verb+.icl+ module a file \verb+.abc+ will 
be created in the corresponding Clean System Files folder. This code is input for 
the Code Generator to build native object files.

\section{The Code Generator and Linker}

The Code Generator takes sequential abc-code as input (an \verb+.abc+ file) and produces
an object file in the Clean System Files folder. The Linker links all object 
files of the project together and builds an application.

By default, the Code Generator generates code for the machine that you are 
working on.
When you are developing your program you should use the
Check Indices option.
This will add checks for out-of-bound array operations.
After you have convinced yourself that your program doesn't violate this condition, you can remove the checks for optimal speed.

Heap and stack space must be allocated to the application. Also other options 
can be set (see \verb+Project Options+ in the \verb+Project+ menu). These settings take effect the 
next time the application is linked.
On demand, the Code Generator can produce assembly files. To produce the 
assembly code choose Generate Assembly in the \verb+Module+ menu. The 
assembly file is generated in the same directory as the Clean \verb+.icl+ module with 
the \verb+.s+ extension and is automatically opened in the editor.

\section{The File Menu}
\screen{menu_file}

\begin{menu}
\item   [New File...]
A standard New File dialogue is diplayed allowing you to specify which file is to
be created. A window is opened on this new file which can then be edited with the
text editor.

\item	[New Project...]
A standard New File dialogue is displayed allowing you to specify where the new project
is to be created. Initially the associated main module is set to be the module
displayed in the active window when \verb+New Project+ was selected. If no module was
active the project does not have a main module yet and one needs to be set by activating
a module window and selecting \verb+Set Mein Module+ from the \verb+Project+ menu.

\item   [Open...]
Open a window on an existing file or project.

\item   [Save]
Save the contents of the active (front) window to the corresponding file.

\item   [Save As...]
Save the contents of the active (front) window to the indicated file. A standard 
New File dialogue appears, asking you to specify the file's name and location.

\item   [Save All]
This command saves the contents of all open editor windows.

\item   [Revert]
Revert to the last saved version of the file corresponding to the active (front) 
window.

\item   [Close]
Closes the active (front) window.
A dialogue will appear when a text window is closed with contents
that need to be saved.

\item   [Close All]
This command closes all open windows.

\item   [Open Other]
This command will open the corresponding \verb+.dcl+ file if the front window is a \verb+.icl+
file or vice versa.

\item   [Open Definition]
When a portion of text is selected (for example the text `selection') the command
is enabled and changes to \verb+Open selection.dcl+. When this command is chosen
it will try to open the corresponding \verb+.dcl+ file. It can be used to quickly open
imported modules: just select 
the name of the file in the \keyword{import} statement and choose this command. For an 
easy way to locate the definitions of imported functions and types see \verb+Find Definition+
and \verb+Find Implementation+ under the \verb+Search+ menu.
When no module name is selected, you can specify the module name in a dialogue box.

\item   [Open Implementation]
This command works the same as the Open Definition command, but now the 
implementation module (\verb+.icl+ file) will be opened.

\item	[Print Setup]
Display the standard Print Setup dialogue for setting default printer options.

\item	[Print...]
Display the standard Print dialogue for printing the contents of the currently active
editor window.

\item	[Print All]
Prints all open editor windows.

\item	[Recent Files]
This is a submenu containing the twelve most recently opened files. Selecting one of these
reopens the corresponding file if it was closed or otherwise makes it the active window.

\item	[Recent Projects]
This is a submenu containing the twelve most recently opened projects. Selecting one of these
reopens the corresponding project after saving the currently active project.

\item   [Quit]
Quit the Clean Integrated Development Environment.
\end{menu}

\section{The Edit Menu}
\screen{menu_edit}

\begin{menu}
\item   [Undo]
Undo the last change made in the text of the front window.

\item   [Cut]
Cuts the text currently selected in the front window to the clipboard.

\item   [Copy]
Copies the text currently selected in the front window to the clipboard.

\item   [Paste]
Insert the contents of the clipboard at the current cursor location or replace the 
currently selected text in the front window by the contents of the clipboard.

\item   [Clear]
Clear the text currently selected in the front window.

\item   [Shift Left / Shift Right]
With these commands you can (un)indent the selected text.
{\it Warning: undo is not implemented for these commands.}

\item   [Balance]
This commands selects the smallest balanced piece of text that surrounds the 
current cursor position or currently selected text.
This feature can be used to find corresponding open and close symbols in 
Clean programs.
The following symbols are balanced: Parentheses: `(' and `)' to check expressions,
Brackets: `[' and `]' to check lists, Curly braces: `\{' and `\}'  to check annotations
or record definitions.

\item   [Select All]
Select all text in the front window.

\item	[Detab]
Replaces all tabs in the front window by the correct number of spaces.

\item	[Add Prefix / Remove Prefix / Change Prefix...]
These three together are a generalisation of the \verb+Shift Left+ and \verb+Shift Right+
commands. They allow the adding and removing of any prefix to the lines in the current selection.
Useful for example for block commenting some code. Note that the \verb+Change Prefix+ dialogue performs
some mangling on the prefix string, that is, it interprets \verb+\\+ as \verb+\+ and \verb+\t+ as a tab.
Other \verb+\+{\it x} combinations are currently just taken as literals.
{\it Warning: undo is not implemented for these commands.}

\item   [Next Window]
Activates the window directly under the active window and brings it to the front. 
The active window is moved to the bottom of the stack of windows that are 
currently open in the Clean Integrated Development Environment.

\item   [Show Clipboard / Hide Clipboard]
This command opens a window on the Clipboard or closes this window when it 
is already open.

\item[List Definitions...]
This opens a popup menu with the global definitions found in the currently active editor window.
The list is orted by definition type and alphabetically. Selecting an entry moves the cursor to the
appropriate line.
\end{menu}

\section{The Search Menu}
\screen{menu_search}
\begin{menu}
\item   [Find...]
A dialogue appears in which one can define a text to search for (in the Find box) 
and the text to replace the found text with (in the Replace box).

By default the Find and Replace starts from the current cursor position or 
currently selected text to the end of the text. On demand the Find and Replace 
goes backwards to the beginning of the text. When Wrap Around is chosen the 
Find and Replace does not stop at the beginning and end of the text in the 
active window. When Ignore Case is chosen, no distinction is made between 
upper and lower case characters in the text to search for. When the Match 
Words option is checked only whole words will be selected. When the text to 
search for appears as a part of another word it will be ignored. Whole words 
are delimited by outline characters (spaces, tabs, (semi)colons, quotes etcetera).
When Use Reg Exp is chosen the Find and Replace strings wil be interpreted as
regular expression as detailed in~\inref{regular_expressions}.

\item   [Find Next]
This will find the next piece of text as defined in the Find 
dialogue. When the \verb+Shift+ key is held down the command finds the previous
piece of text as defined in the Find dialogue.

\item   [Find Selection]
The selected piece of text will be entered into the Find box and the 
next occurrence of the selection will be found. When \verb+Shift+ is pressed the 
previous occurrence will be found. 

\item   [Replace \& Find Again]
Replace selected text with the text as defined in the Replace box and find the next 
piece of text as defined in the Find box. When the \verb+Shift+ key is being held 
down the previous occurrence will be found.

\item   [Search Identifier]
If there is a selection in the active window then this command wil attempt to find
this selection in all files in the current project. If there is no selection or the
\verb+Shift+ key was pressed when selecting this command a dialogue appears allowing
you to enter the search string and set the search method.

Find Identifiers locates all occurrences of an identifier. You can use this for 
example to change the name of an identifier in all the modules of a program. 
The search results are displayed in the Search Results window. You can use \verb+Goto Next Search Result+
(\verb+Ctrl-,+)
to jump to the next occurrence.

If Be Verbose is chosen the file currently being searched is listed.

\item   [Search Definition]
When a Clean symbol (function, type or constructor) is selected this command 
will try to find the definition of this symbol in its definition module. Pressing 
Ctrl and double-clicking on the name has the same effect. Macros and 
functions that have no type specification may not be found. When your Clean 
modules have a (very) unusual layout the search algorithm may also not be 
able to find the definitions of Clean symbols.

\item   [Search Implementation]
This command does the same as \verb+Find Definition+, but now the implementation 
modules will be searched for the symbol. Holding down \verb+Shift+ and \verb+Ctrl+ and 
double-clicking on the name has the same effect as this command.

\item	[Goto Next Search Result]
Move to the location of the next entry in the Search Results window.

\item   [Goto Next Error]
This command selects the next error message in the Errors window and also 
selects the corresponding text line in which the error occurred. When an error 
occurs in a file which is not opened a window on this file will be opened 
automatically. This command only has an effect when the Errors \& Warnings window is 
open.

\item   [Goto Cursor]
Scroll front window to show the text at the current cursor position.

\item   [Goto Line...]
Select the line with the indicated line number. Define the desired line number 
in dialogue box. The lines in a text window are numbered starting from 1. The line 
with the indicated line number is selected. For other ways of moving the cursor 
around and selecting lines see~\inref{text_editor}.

\item	[Goto Selection...]
Define the desired selection in the dialogue box. The given selection will be highlighted and
displayed in the text window.

\end{menu}

\section{The Project Menu}
\screen{menu_project}
\begin{menu}
\item   [Bring Up to Date]
This command will bring the project up-to-date.

This means that, starting from the main module all implementations that are 
inconsistent will be recompiled . Also the inconsistent object files 
are regenerated and a new application is built.

Inconsistencies can arise when implementation modules are modified or when 
implementation modules have to be recompiled because they are depending on 
modified modules.

After an update all files are consistent. Launching of an application can now be 
done without recompilations.

\item   [Update and Run]
This command first brings a Clean program up-to-date and then 
launches it.

The \verb+Update and Run+ command will perform the same actions as the \verb+Bring Up To Date+
command explained above. In addition the generated application is launched. 
The system will warn you when there is not enough free memory left to launch 
the generated application.

\item	[Run]
If you are convinced that the application already exists and is up-to-date then you can use
the \verb+Run+ command to immediately launch it without first performing the consistency checks
performed by the \verb+Update and Run+ command described above.

\item   [Set Main Module]
Set the main module of the currently active project to the implementation 
module displayed in the front window. Note that this clears all the found dependancy information
for the current project and all the associated compiler and editor settings.

\item   [Project Options]
Here the project options for the current project can be set, see~\inref{project_options}.

\item	[Show Heap Profile]
This launches the \verb+Show Heap Profile+ application and attempts to open the heap profile
for the current project. This wil only succeed if a heap profile has been generated for the
current project. That is if heap profiling has been selected in the \verb+Profiling+ pane of
the \verb+Project Options+ of the current project, and then the generated application has been
executed.

\item	[Show Time Profile]
This launches the \verb+Show Time Profile+ application and attempts to open the time profile
for the current project. This wil only succeed if a time profile has been generated for the
current project. That is if time profiling has been selected in the \verb+Profiling+ pane of
the \verb+Project Options+ of the current project, and then the generated application has been
executed.

\item[Theorem Prover Project]
This launches the Clean Theorem Prover and attempts to open a proof project for the currenlty
active project in the IDE.

\item[Theorem Prover Module]
This command is currently disabled.
\end{menu}

\section{The Module Menu}
\screen{menu_module}
\begin{menu}
\item   [Compile]
Compile the Clean implementation module (\verb+name.icl+) displayed in 
the active window.
Error messages will appear when inconsistencies are found. With \verb+Find Error+ in 
the \verb+Search+ menu the line in which the error was found is selected.
When no errors are found, type information is displayed in the Types window 
when this option was chosen (see~\inref{module_options}).

A file \verb+name.abc+ will be created in the folder Clean System Files. This folder will 
also be created if it does not yet exist in the same folder where 
\verb+name.icl+ and \verb+name.dcl+ are located.

\item   [Check Syntax]
Check the Clean implementation module (\verb+name.icl+) displayed in the 
active window.
Check Syntax does the same as compile, but now no \verb+.abc+ file is generated.

\item   [Generate Assembly]
Generate an assembly file corresponding to the Clean implementation
module (\verb+name.icl+) displayed in the active window. The assembly file is 
placed in the same folder as the icl file and has the extension \verb+.s+ (for example 
\verb+name.s+).

The assembly file is generated by the Code Generator taking a \verb+name.abc+ file as 
input. This file is generated by the Clean compiler. If the \verb+.abc+ file is not present 
or inconsistent the Generate Assembly command will invoke a compilation of 
the implementation module.

\item	[Module Options]
This displays a dialogue where a number of options can be set for the currently active module,
see~\inref{module_options}.

\end{menu}

\section{The Defaults Menu}
\screen{menu_defaults}
\begin{menu}
\item[Module Defaults]
A dialogue appears to set the default module options for new modules added to a project,
see~\inref{module_options}.

\item	[Project Defaults]
In this dialogue the default Project Options for new projects can be set, note that paths
and extra link items cannot be set by default. If this is required then add them to an
environment. See~\inref{project_options} and \inref{environment_options}.
\item	[Window Settings]
The followingmenu entries are placed under the submenu Window Settings:
\begin{description}
	\item	[Editor Colours...]
	Here you can set the colours used for syntax colouring the editor windows.
	\item   [Editor Settings...]
	This dialogue allows you to set a new text font and text size for editor
	windows. Also the tab width can be redefined. With the Auto Indent 
	checkbox the auto indent facility of the text editor can be switched on and off.
	\item	[Project Window]
	Here you can set some font and colour information for the project window.
	\item	[Types Window]
	This allows you to set the font and syntax colouring information for the Types window.
	\item[Console Window]
	This allows you to to set the font and syntax colouring information for the Console window.
	\item	[Search Window]
	Here you can set the font and colours used in the Search Results window.
	\item	[Error Window]
	Here you can set the font and colours used in the Errors \& Warnings window.
\end{description}

\item	[Key Mapping...]
In this dialogue the key mappings for the editor can be set, see~\inref{key_mappings}.

\item	[IDE Options...]
In this dialogue some global preferences for the Clean IDE can
be set. See \inref{ide_options}.
\end{menu}

\section{The Environment Menu}
\screen{menu_environment}
\begin{menu}
\item	[Edit Current...] Allows you to edit the settings for the currently active environment. See~\inref{environment_options}.
\item	[New...] Displays a dialogue asking for the name of the environment to be added, and
	then displays the options dialogue.
\item	[Edit List...] Allows you to edit the list of known environments.
\end{menu}
Below these comes a list with the currently available environments with the currently
active environment indicated by a check mark. You can change the active environment
by selecting the desired environment from the list.

\section{The Help Menu}
\screen{menu_help}
\begin{menu}
\item	[About...]
Displays an about window for the Clean IDE. The current version number can be found here,
please mention this when reporting bugs.

\item   [Help]
This submenu displays the documents found in the |Help| directory. Selecting a help item
opens the associated document in an appropriate application. E.g. Acrobat Reader for this manual.

\end{menu}

\section{The Window Menu}
\screen{menu_window}
The Window menu displays a platform dependant number of
options for arranging the open windows and switching between them.

%	\section{The Time Profiler}
%	Under construction.
%	
%	\section{The Heap Profiler}
%	Under construction.

\appendix
\section{Keymappings}\label{key_mappings}
The editor also supports (user modifiable) key mappings.
The preinstalled default mapping is as follows:

\begin{longtable}{|l|l|l|}
\hline
\light	SHIFT + ALT + LeftKey		& Select	& WordLeft				\\
\light	CTRL + SHIFT + LeftKey		& Select	& StartOfLine				\\
\light	SHIFT + LeftKey				& Select	& CharLeft			\\
\light	ALT + LeftKey				& Move		& WordLeft			\\
\light	CTRL + LeftKey				& Move		& StartOfLine		\\
\light	LeftKey						& Move		& CharLeft		\\
\hline
\dark	ALT + SHIFT + RightKey		& Select	& WordRight				\\
\dark	SHIFT + CTRL + RightKey		& Select	& EndOfLine			\\
\dark	SHIFT + RightKey			& Select	& CharRight				\\
\dark	ALT + RightKey				& Move		& WordRight		\\
\dark	CTRL + RightKey				& Move		& EndOfLine		\\
\dark	RightKey					& Move		& CharRight			\\
\hline
\light	ALT + SHIFT + UpKey			& Select	& PageUp			\\
\light	CTRL + SHIFT + UpKey		& Select	& StartOfText				\\
\light	SHIFT + UpKey				& Select	& LineUp				\\
\light	ALT + UpKey					& Move		& PageUp			\\
\light	CTRL + UpKey				& Move		& StartOfText			\\
\light	UpKey						& Move		& LineUp			\\
\hline
\dark	ALT + SHIFT + DownKey		& Select	& PageDown				\\
\dark	CTRL + SHIFT + DownKey		& Select	& EndOfText			\\
\dark	SHIFT + DownKey				& Select	& LineDown			\\
\dark	ALT + DownKey				& Move		& PageDown		\\
\dark	CTRL + DownKey				& Move		& EndOfText		\\
\dark	DownKey						& Move		& LineDown		\\
\hline
\light	BeginKey					& Scroll	& StartOfText				\\
\light	EndKey						& Scroll	& EndOfText			\\
\light	PgUpKey						& Scroll	& PageUp			\\
\light	PgDownKey					& Scroll	& PageDown			\\
\hline
\dark	CTRL + DeleteKey			& Remove	& StartOfLine			\\
\dark	ALT + DeleteKey				& Remove	& WordRight		\\
\dark	DeleteKey					& Remove	& CharRight		\\
\hline
\light	CTRL + BackSpaceKey			& Remove	& EndOfLine		\\
\light	ALT + BackSpaceKey			& Remove	& WordLeft		\\
\light	BackSpaceKey				& Remove	& CharLeft			\\
\hline
\end{longtable}

A customized mapping can be made by defining a mapping using the \verb+Key Mapping+ dialogue in the
\verb+Defaults+ menu and then saving it. Any saved keymapping can later be reloaded from the same
dialogue. If the keymapping is then installed in the ``Config'' directory
under the name \verb+default.km+ then in future the IDE will use that
as its default key mapping rather than the preinstalled one given above.
In the distribution an alternative (Windows) keymapping is provided under the name \verb+windows.km+

\newcommand{\entrylabel}[1]{\mbox{\textsf{#1}}\hfil}
\newcommand{\textbackprime}{\raisebox{3pt}{\mbox{$\backprime$}}}

\section{Command shortcuts}
Alphabetic list of (hopefully) all command key combinations in the Clean IDE:
\begin{multicols}{2}
\begin{description}
\setlength{\itemsep}{0pt}
\renewcommand{\makelabel}{\entrylabel}
%\item[Ctrl-`] show pop-up of top level definitions
%\item[Ctrl-$\backprime$] show pop-up of top level definitions
\item[Ctrl-\textbackprime] show pop-up of top level definitions
\item[Ctrl-/] open other
\item[Ctrl-\textbackslash] next window
%\item[Ctrl-|\|] next window
\item[Ctrl-Shift-\textbackslash] prev window
\item[Ctrl-=] find identifier in project
\item[Ctrl-Shift-=] show find identifier dialogue
\item[Ctrl-[] shift left
\item[{Ctrl-]}] shift right
\item[Ctrl-,] next search result
\item[Ctrl-Shift-,] previous search result
\item[Ctrl-A] select all
\item[Ctrl-B] balance
\item[Ctrl-C] copy
\item[Ctrl-D] open definition module
\item[Ctrl-Shift-D] show open definition module dialogue
\item[Ctrl-E] find next error
\item[Ctrl-Shift-E] find previous error
\item[Ctrl-F] open find dialogue
\item[Ctrl-Shift-F] open find dialogue with backwards option preselected
\item[Ctrl-G] repeat previous find command
\item[Ctrl-Shift-G] repeat previous find command in reverse direction
\item[Ctrl-H] find current selection
\item[Ctrl-Shift-H] find current selection in reverse direction
\item[Ctrl-I] open implementation module
\item[Ctrl-Shift-I] show open implementation module dialogue
\item[Ctrl-J] open module options dialogue
\item[Ctrl-K] compile current module
\item[Ctrl-L] find definition of current selection
\item[Ctrl-Shift-L] show find definition dialogue
\item[Ctrl-M] find implementation of current selection
\item[Ctrl-Shift-M] show find implementation dialogue
\item[Ctrl-N] new editor file
\item[Ctrl-Shift-N] new project file
\item[Ctrl-O] open file or project
\item[Ctrl-Shift-O] open file as text,
	same as above but always open as text
\item[Ctrl-P] print window
\item[Ctrl-Shift-P] print all editor windows
\item[Ctrl-Q] quit
\item[Ctrl-R] update and run
\item[Ctrl-Shift-R] run
\item[Ctrl-S] save editor window
\item[Ctrl-Shift-S] save all editor windows
\item[Ctrl-Alt-S] save as
\item[Ctrl-Alt-Shift-S] save copy as
\item[Ctrl-T] replace and find again
\item[Ctrl-U] bring up to date
\item[Ctrl-Shift-U] force update,
	recompiles all modules
\item[Ctrl-V] paste
\item[Ctrl-W] close window
\item[Ctrl-Shift-W] close all editor windows
\item[Ctrl-X] cut
\item[Ctrl-Y] check syntax
\item[Ctrl-Z] undo
\end{description}
\end{multicols}

\section{Regular Expressions}\label{regular_expressions}
In addition to its regular
searching method the IDE has some support for searching in
files using regular expressions. This feature is still
under development, and is still rather slow...

The syntax for regular expressions to be found is as follows:

\begin{tabbing}
exprs \= = \=beg exprs end \=:	\kill
expr	\> = \> beg exprs end			\>: search string indicated by regular expression\\
beg		\> = \> \verb+^+ 				\>: search for an exprs at beginning of a line\\
		\> || \> \verb+|+				\>: search for exprs at the beginning of a word\\
		\> || \>							\>: search for exprs starting anywhere\\
end		\> = \> \verb+^+ 				\>: search for an exprs ending at the end of a line\\
		\> || \> \verb+|+				\>: search for exprs ending at a word\\
		\> || \>							\>: search for an exprs ending anywhere\\
exprs	\> = \> RE						\>: search for indicated expression\\
		\> || \> RE expr					\>: search for sequence of indicated expressions\\
exprs	\> = \> \verb+{+expr\verb+}+	\>: search for expr and remember the corresponding matching expression as substring\\
RE		\> = \> unit					\>: search for unit\\
		\> || \> unit \verb+*+			\>: search for zero or more occurrences of unit\\
		\> || \> unit \verb!+!			\>: search for one  or more occurrences of unit\\
unit	\> = \> $c$						\>: search for the indicated character\\
		\> || \> \verb+(+exprs\verb+)+	\>: turn exprs into unit\\
char	\> = \> $c$						\>: search for the specified character, if it is not used as special RE character\\
		\> || \> \verb+#+$c$				\>: search for the specified character, even if it is a special RE character\\
		\> || \> \verb+.+				\>: search for any character\\
		\> || \> \verb+@+				\>: search for any letter\\
		\> || \> \verb+~+				\>: search for any non-letter\\
		\> || \> \verb+[+alt\verb+]+		\>: search for any of the specified characters\\
alt		\> = \> $c$						\>: this character \\
		\> || \> $c1$$c2	$				\>: or any character in this sequence\\
		\> || \> $c1$\verb+..+$c2$		\>:	or any character in this interval
\end{tabbing}

\noindent
%where the special RE characters are: \{ \} * + ( ) \# . @ \textasciitilde [ ] 
where the special RE characters are: \verb?{ } * + ( ) # . @ ~ [ ]?

\noindent
And the replacement expression can be specified as follows:

\begin{tabbing}
exprs \= = \=beg exprs end \=:	\kill
rexpr	\> = \> exprs			\> : replace string found by exprs\\
		\> || \> exprs rexpr		\> : replace string found by sequence of exprs\\
exprs	\> = \> $c$				\> : replace string found by the indicated character, if no control character\\
		\> || \> \verb+#+$c$		\> : replace string found by the specified character, even if it is a control character. \\
		\> || \> \verb+&+		\> : replace string found by itself\\
		\> || \> \verb+$+$n$		\> : replace string found by the $n$-th substring indicated in the search expression, $0 \leq n \leq 9$\\\
		\>   \> 				\> : replaced by empty string if the $n$-th substring does not exist
\end{tabbing}

\noindent
where the replacement expression control characters are: \verb+& $ #+

\section{Bugs}

If you have bugs to report, please send us a short description with a tiny
example via electronic mail (\url{mailto: clean@cs.kun.nl}) and we will try to fix it as
soon as possible.

\section{Authors \& Credits}

Clean and the Clean System are a spinoff of the 
research performed by the research group on functional programming 
languages, Computing Science Institute, at the University of Nijmegen under the 
supervision of Rinus Plasmeijer. To see who is currently responsible for which part
of Clean and the Clean System have a look at the Clean web pages.

Special thanks to the following people:
Christ Aarts, Steffen van Bakel, Erik Barendsen, Henk Barendregt, Pieter Hartel, 
Hans Koetsier, Pieter Koopman,  Ken McDonald, Ronan Sleep and all the Clean 
users who helped us to get a better system.

Many thanks to the following sponsors:
\begin{itemize}
\item the Dutch Foundation for Applied Technical Sciences (STW);
\item the Dutch Foundation for Scientific Research (NWO);
\item the International Academic Centre for Informatics (IACI);
\item Kropman B.V., Installation Techniques, Nijmegen, The Netherlands.
\item Hitachi Advanced Research Laboratories, Japan;
\item the Dutch Ministry of Science and Education (the Parallel Reduction Machine 
project (1984-1987)) who initiated the Clean research;
\item Esprit Basic Research Action (project 3074, SemaGraph: the Semantics and 
Pragmatics of Graph Rewriting (1989-1991));
\item Esprit Basic Research Action (SemaGraph II working group 3646 (1992-
1995));
\item Esprit Parallel Computing Action (project 4106, (1990-1991));
\item Esprit II (TIP-M project area II.3.2, Tropics: TRansparent Object-oriented 
Parallel Information Computing System (1989-1990)).
\end{itemize}

%\printindex
\end{document}
